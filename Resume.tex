%%%%%%%%%%%%%%%%%%%%%%%%%%%%%%%%%%%%%%%%%%%%%%%%%%%%%%%%%%%%
%
% author: Akhil Reddy
%
%%%%%%%%%%%%%%%%%%%%%%%%%%%%%%%%%%%%%%%%%%%%%%%%%%%%%%%%%%%%

\documentclass{article}

\usepackage{fullpage}
\usepackage[hidelinks]{hyperref}
\usepackage{amsmath}
\usepackage{amssymb}
\usepackage[T1]{fontenc}
\usepackage{fancyhdr}
\usepackage{lastpage}
\usepackage{graphicx}
% \usepackage{hyperref}
\usepackage{fontawesome}
\usepackage{setspace}
\usepackage[margin=0.5in]{geometry}

\newcommand\blfootnote[1]{%
  \begingroup
  \renewcommand\thefootnote{}\footnote{#1}%
  \addtocounter{footnote}{-1}%
  \endgroup
}

% DEFINITIONS FOR RESUME

\textheight=10in
\pagestyle{fancy}
\raggedright
\fancyhf{}
\renewcommand{\headrulewidth}{0pt}

\setlength{\hoffset}{-2pt}
\setlength{\footskip}{20pt}

\def\bull{\vrule height 0.8ex width .7ex depth -.1ex }

\newcommand{\contact}[3]{
\vspace*{3pt}
\begin{center}
{\LARGE \scshape {#1}}\\
\vspace{5pt}
#2 
\vspace{2pt}
#3
\end{center}
\vspace*{-8pt}
}

\newcommand{\header}[1]{{
\hspace*{0pt}\vspace*{6pt} \textsc{#1}} \vspace*{-6pt} 
\lineunder
}

\newcommand{\lineunder}{
\vspace*{-8pt} \\ \hspace*{-3pt} 
\hrulefill \\
}

\newcommand{\content}{
\vspace*{2pt}
}

\newcommand{\school}[5]{\vspace*{2pt}% 
\textbf{#1} \labelitemi #2 \hfill #3 \\ #4 \hfill #5
\vspace*{5pt}
}

\newcommand{\college}[7]{
\textbf{#1} \labelitemi \textbf{#2} \hfill #3 \\ #4 \hfill #7 \\ #5 \\ #6 \vspace*{5pt}
}

\newcommand{\employer}[4]{{
\vspace*{2pt}%
\textbf{#1} #2 \hfill #3\\ #4 \vspace*{2pt}}
}

\newcommand{\project}[3]{{
\vspace*{2pt}% 
\textbf{#1} #2 \hfill #3\vspace*{2pt}}
}

\renewcommand{\labelitemi}{
$\vcenter{\hbox{\tiny$\bullet$}}$\hspace*{3pt}
}

\renewcommand{\labelitemii}{
$\vcenter{\hbox{\tiny$\bullet$}}$\hspace*{-3pt}
}

\newenvironment{bullet-list-major}{
\begin{list}{\labelitemii}{\setlength\leftmargin{9pt} 
\topsep 0pt \itemsep -2pt}}{\vspace*{4pt}\end{list}
}

\newenvironment{bullet-list-minor}{
\begin{list}{\labelitemii}{\setlength\leftmargin{15pt} 
\topsep 0pt \itemsep -2pt}}{\vspace*{4pt}\end{list}
}

% END RESUME DEFINITIONS

\begin{document}

\small
\smallskip
\vspace*{-44pt}

\contact{\textbf{Akhil Reddy}}
{\faPhone\ (732) 666-7264
\labelitemi \faEnvelope\ \href{mailto:avr54@scarletmail.rutgers.edu}{akhil.reddy@rutgers.edu}
\labelitemi \faLinkedin\ \href{https://www.linkedin.com/in/akhilvreddy/}{linkedin.com/in/akhilvreddy}
\labelitemi \faGithub\ \href{https://github.com/akhilvreddy}{github.com/akhilvreddy}%
\labelitemi \faLink\ \href{https://akhilvreddy.github.io}{akhilreddy.me}

}

\vspace{4pt}
\header{Education}
    \college{Rutgers University}{New Brunswick, NJ}{September 2020 -- May 2024}
    {\textit{Bachelor of Science:}  Computer Engineering}
    {\textit{Minors:} Physics \& Mathematics}{} 
    {GPA: 3.74/4.0}

    
\vspace*{4pt}%
\header{\textbf{Highlighted Coursework}}
    \begin{bullet-list-major}
    \item \textbf{Computer Engineering:} Data Structures, Computer Architecture, Circuits I-II, Digital Logic Design, Network Security, Probability \& Random Processes, Discrete Math, Computation for Engineers, Software Engineering, Linear Systems \& Signals
    \item \textbf{Physics \& Math:} Classical Mechanics I-II, Electromagnetism I-II, Modern Physics, Quantum Mechanics, Thermal Physics, Computer Experimentation, Linear Algebra, Multivariable Calculus, Partial Differential Equations, Classical Physics Lab I-II
    \end{bullet-list-major}

\vspace*{4pt}%
\header{Skills}
    \begin{bullet-list-major}
    \item \textbf{Programming languages:} Java, Python, Web (HTML/CSS/JavaScript/Typescript), C, Verilog, R
    \item \textbf{Software \& Frameworks:} Git, GitHub, Android Studio, React, Matlab, Mathematica, Maple, Arduino, SQL, AWS, Linux, Excel
    
    \end{bullet-list-major}
    
\vspace*{4pt}%
\header{Experience}
    \employer{Jalali Lab}
    { -- \href{https://github.com/akhilvreddy/Autoencoder}{GitHub \faGithub\ }}{May 2022 -- Present}{Machine Learning Researcher}
    \begin{bullet-list-minor}
	\item Designed algorithms that combatted the effect of speckle noise in ISAR imaging by using compression based sensing. Trained my neural network using PyTorch and Pandas with a data set including images of lungs with pneumonia.
	\item Reconstructed distorted images using the neural network's ability to identify and sharpen images by passing them through the trained neural network. Was able to achieve 90\% resolution of original, no-noise image from this method.  
    \end{bullet-list-minor}

    \employer{Javanmard Lab}
    {-- \href{https://github.com/akhilvreddy/Coral-Reef}{GitHub \faGithub\ }}{Jan 2022 -- July 2022}{Android Application Developer}
	\begin{bullet-list-minor}
	\item Analyzed coral reef health by designing an android application that can detect and read RGB values from 70+ coral. Applied border detection in java by designing an algorithm that checks pixel shade of the coral against the background.
	\item Worked with YOLOv4 and trained AI to detect specific kind of coral using TensorFlow. Implemented an algorithm that analyzed how these coral react to different chemicals by using off-delay timers and 6 color sensors to further analyze health.
    \end{bullet-list-minor}
    
    \employer{Chandrasekhar Star Modeling}
    {}{August 2021 -- December 2021}{Undergraduate Research Assistant}
	\begin{bullet-list-minor}
	\item Investigated star temperature data to fit known differential equations \& reworked. Utilized Mathematica to do variable based calculations to simplify our equations and used C to do number crunching.
	\item Proved correlation between data from Y. Eriguchi’s paper to initially proposed equations by Chandrashekar in 1939 by showing that both of the numerical solutions aligned by 85\%. 
    \end{bullet-list-minor}


\vspace*{4pt}%
\header{Projects}
    \project{NBA Neural Network}{-- Java, Python, Pandas, PyTorch -- \href{https://github.com/akhilvreddy/NBANeuralNetwork}{GitHub \faGithub\ }}{July 2022}
    \begin{bullet-list-minor}
    \item Engineered a NBA odds predictor by training a neural network which resulted in outcomes that are comparable to highly reputed odd-makers such as DraftKings about 80\% of the time.
    \item Utilized PyTorch to analyze statistics of 200+ past games and sifted through data provided by the NBA API using excel. 
    \item Produced over/under odds by calculating and assigning corresponding weights to game statistics and included functionality to input live game scores, which would make the odds change instantaneously. 
    \end{bullet-list-minor}

    \project{Quantum Wavefunction Analysis}{--  Mathematica, Jupyter, Numpy -- \href{https://github.com/akhilvreddy/Quantum-Wavefunction-Analysis}{GitHub \faGithub\ }}{May 2022}
	\begin{bullet-list-minor}
	\item Implemented Schrodinger's equation with python for a single particle in varying potentials and calculated 5+ wavefunctions for corresponding particles using fuctions from numpy and numba. 
	\item Plotted resulting potentials and wavefunctions in Jupyter using matplotlib and created a gif of the wavefunction evolution over time using animation by overlapping two solutions, which was 50\% faster than using an online solver.
    \end{bullet-list-minor}



\vspace*{4pt}%
\header{Leadership}
    \employer{Society of Physics Students, Rutgers Chapter}
    {}{September 2020 -- Present}{Treasurer}
	\begin{bullet-list-minor}
    \item Organized 10+ events such as panels, research talks etc to garner interest in physics for middle and high school students and increased club engagement and participation by 30+ members.
    \item Track \$3,500 budget for events, draft financial summaries, submit fiscal end of semester packets, and manage bank account by monitoring expenses for the school year.
    \end{bullet-list-minor}
    \employer{First Year Integration}
    {}{September 2021 -- Present}{Program Leader}
    \begin{bullet-list-minor}
    \item Mentored 5 first-year students to promote success in engineering by hosting 1-1 meetings and planning engaging events. Assisted with the transition between high school and undergraduate life by giving them advice about school, life, and plans post-graduation.  
    \end{bullet-list-minor}

\end{document}
